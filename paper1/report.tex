\documentclass[sigconf]{acmart}

\usepackage{hyperref}

\usepackage{endfloat}
\renewcommand{\efloatseparator}{\mbox{}} % no new page between figures

\usepackage{booktabs} % For formal tables

\settopmatter{printacmref=false} % Removes citation information below abstract
\renewcommand\footnotetextcopyrightpermission[1]{} % removes footnote with conference information in first column
\pagestyle{plain} % removes running headers

\begin{document}
\title{Big Data and Artificial Neural Networks}



\author{Bharat Mallala}
\affiliation{%
  \institution{Indiana University}
  \streetaddress{Smith Research Center\newline
                 2805 E. 10th St, Suite 150}
  \city{Bloomington, IN 47408} 
  \country{USA}}
\email{bmallala@iu.edu}


% The default list of authors is too long for headers}
\renewcommand{\shortauthors}{B. Trovato et al.}


\begin{abstract}
Big data is often referred as a problem of dealing with large data sets. With the advancements in computational science and the recent evolution of Artificial Intelligence(AI) and Machine Learning, huge volumes of data is being generated every day. Simultaneously the computational resources needed to process and analyze this data is trying to catch up with the rapidly growing data and for the most part have succeeded. In today's world there is a large dependency on Neural networks for dealing with problems in AI and Data analysis. This paper addresses how Big data and its applications can be used to addresses various issues that arise with Artificial Neural Networks(ANN). 
\end{abstract}

\keywords{Artificial Neural Networks, Machine Learning, Artificial Intelligence, Data Analysis, Perceptron.}


\maketitle

\section{Introduction}

Artificial Neural Networks are often referred as a Multi-layer Neural Network where each node in the network is a Perceptron. It often mics the human brain, i.e. it works in a similar fashion. Advancements in ANN's and its ability to solve complex problem at a relatively faster rate than the traditional approaches have made it the top choice for solving the usually NP-hard AI problems. "Visual analysis systems will all require a neural network behind them, and that involves a lot of compute power"\cite{1} quoted Anderson. This explains the efficiency of Neural networks in solving problems and analysis. ANN's take a series of inputs from the users and map them accordingly to find reasonable patterns in data.

Certainly with these advancements comes huge volumes of data which needs to processed efficiently. This is where Big data comes into picture with its ability to store and process large data sets of any kind for example audio, video, images,text etc in relatively less time. "s Big Data Analytics is an effective and capable way to, not only work with these data, but understand its meaning, providing inputs for assertive analysis and predictive actions."\cite{2} quotes Victor P Barros in paper.

Artificial Neural Networks usually consists of three primary layers, input layer, output layer, hidden layer. There may be multiple layers of perceptrons within the hidden layer. From the figure 1 we can see the three layers of the ANN. The input layer takes in the input as a set of features and its corresponding weights and the output layer returns a predicted value. All the calculations are done in the hidden layer. The ANN's typically use the feed forward algorithm combined with back propagation for its calculation. The network initially feeds forward to the very end and generates an output from the initial set of features and weights. It then back propagates using Gradient descent and recalculates the wights for each iteration. The algorithm finally stops of the difference in weights form one iteration to the other is not greater than a pre defined threshold. We then test this on the training set and evaluate the performance of the network.


\section{Conclusions}

This is my Conlusion


\begin{acks}

  The authors would like to thank Dr. Gregor von Laszewski for all the help he has provided for this paper.
\end{acks}

\bibliographystyle{ACM-Reference-Format}
\bibliography{report} 

\end{document}

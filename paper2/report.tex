\documentclass[sigconf]{acmart}

\usepackage{hyperref}

\usepackage{endfloat}
\renewcommand{\efloatseparator}{\mbox{}} % no new page between figures

\usepackage{booktabs} % For formal tables

\settopmatter{printacmref=false} % Removes citation information below abstract
\renewcommand\footnotetextcopyrightpermission[1]{} % removes footnote with conference information in first column
\pagestyle{plain} % removes running headers

\begin{document}
\title{Big Data and Artificial Intelligence with Computer Vision}

\author{Bharat Mallala}
\affiliation{%
  \institution{Indiana University}
  \city{Bloomington, IN 47408} 
  \country{USA}}
\email{bmallala@iu.edu}






% The default list of authors is too long for headers}
%\renewcommand{\shortauthors}{B. Trovato et al.}


\begin{abstract}
Big data refers to a problem of dealing with huge volumes of data. With the increase in the amount of data generated every day from various fields, it is becoming extremely hard to store and process this data efficiently. Artificial Intelligence is a research field aiming at replicating human work through machines. Computer vision refers to a research area within AI dealing with training computers to recognize certain subjects of interest. With the exponential growth of AI and computer vision in the recent years, there is need to address the big data problem associated with it.
\end{abstract}

\keywords{Artificial Intelligence, Computer vision, Perceptron Deep Learning, Convolutional Neural Networks }


\maketitle

\section{Introduction}
Artificial Intelligence is an aim at replicating human intelligence through machines. The term AI was in existence form the late 1950's during which there was a lot of enthusiasm on its potential during which Alan Turing introduced the Turing test. A lot of research was carried out in AI during the 1960's with the introduction of perceptron theory and its ability to solve problems. There was a major setback for AI in the early 1970's during which Minsky in his book on perceptrons has pointed out the major drawbacks of perceptrons in dealing with complex problems.\cite{}

There has been a consistent growth in AI from the 1990's with the introduction of the statistical approach to problem-solving. With the increase in the use of Big data from 2010, there was a lot of development in the field of AI with many voice assistants, self-driving cars, automated robots etc. Many AI problems which were np-hard previously would now take minutes to solve thanks to recent advancements in big data. Professor Crandall quoted during his lecture quoted that "Problems that seem to require intelligence usually require exploring multiple choices".\cite{}, which is a way of exhibiting intelligence without actually having it. For example for a machine to win a tic-tac-toe game against a human it basically has to explore all the possible choices, it can make from any given state.

Hence we can map an AI problem as a search problem, but it usually requires searching through huge search spaces, this is when it becomes a Big data problem. For example, for a computer t win against a human in chess it needs to search through hundreds of thousands of states. To deal with such huge data, there is a need to apply big data technologies to better store data and efficiently manage it. Ai problems typically involve using both structured and unstructured kind of data in huge volumes. The traditional RDBMS methods have a hard time dealing with unstructured data. This is an area where Big data shines with the ability to deal with both structured and unstructured data efficiently.

"Computer vision is an interdisciplinary field that deals with how computers can be made for gaining high-level understanding from digital images or videos".\cite{wiki} It is an area of research within AI that aims at recognizing subjects in an environment from images or videos. Convolutional Neural Networks shines at image classification from images with minimal prepossessing of the input variables and still manages to obtain better classification. With the exponential rise of AI and especially CNN lately, there is an increased interest in computer vision for researchers. Computer vision involves training the model with huge sets of images and videos which indeed needs to be addressed using big data technologies.

\section{Rethinking the Inception Architecture for Computer Vision}

\subsection{CNN's for Computer vision}

Convolutional Neural Networks(CNN's) is the key to advancements in computer vision in the recent years with its low parameter count and computational efficacy. A CNN has multiple layers with perceptrons in them that help in the information flow from the input to output. A CNN typically has filters that are mapped across the original image to extract useful features from the image. During the training phase of the model, CNN learns the values of the filters. A CNN architecture has mainly two phases convolution phase and pooling phase. In the convolution phase features are extracted from the image using filters. In the pooling phase, the width of the feature map is reduced by applying various techniques. This is done to remove unnecessary data from the features. These stages are iterated till the desired features are obtained from the image.\cite{}   Figure 1 shows a typical CNN architecture.\cite{CNN figure}

\subsection{Design methods}
A lot of design principles need to followed when designing a CNN. With the numerous number of iterations required over the CNN phases, a good design decision can vastly improve the efficiency. Avoiding bottlenecks/cycles in the CNN helps to avoid infinitely

\begin{acks}

  The authors would like to thank Dr. Yuhua Li for providing the
  matlab code of the \textit{BEPS} method.

  The authors would also like to thank the anonymous referees for
  their valuable comments and helpful suggestions. The work is
  supported by the \grantsponsor{GS501100001809}{National Natural
    Science Foundation of
    China}{http://dx.doi.org/10.13039/501100001809} under Grant
  No.:~\grantnum{GS501100001809}{61273304}
  and~\grantnum[http://www.nnsf.cn/youngscientsts]{GS501100001809}{Young
    Scientsts' Support Program}.

\end{acks}

\bibliographystyle{ACM-Reference-Format}
\bibliography{report} 

\end{document}
